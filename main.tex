\documentclass[aps,prd,onecolumn,showpacs,amsmath,amssymb,nofootinbib]{revtex4} \pdfoutput=1
 
\usepackage{amsfonts}
\usepackage{amsmath}
\usepackage{graphicx}
\usepackage{subfigure}
\usepackage{dcolumn}
\usepackage{bm}
\usepackage{booktabs}
\usepackage[utf8]{inputenc}
\usepackage{multirow}
\usepackage{graphicx,graphics,dcolumn,booktabs,bm}
\usepackage{longtable,lscape}
\usepackage{txfonts}
\usepackage{overpic}
\usepackage{amssymb}
\usepackage{indentfirst}
\usepackage{epsfig}
\usepackage{feynmf}   %{feynmp}
\usepackage{epstopdf}   %{feynmp}
\usepackage{slashed}  %for Feynman symbols
\usepackage{color}
\usepackage[section]{placeins}

\usepackage[colorlinks, citecolor=blue,anchorcolor=red,menucolor=red, linkcolor=red,filecolor=red,runcolor=red,urlcolor=blue,frenchlinks=red]{hyperref}



\usepackage{ulem}
\usepackage{physics}


\begin{document}
\title{Effective Theories of the Strong Interaction}

\author{J.~Ho}
\affiliation{Department of Physics and Engineering Physics\\ University of Saskatchewan\\ 
          Saskatoon, SK, S7N 5E2, Canada}

\begin{abstract}
While much interest and resources have been directed towards the TeV energy scales within particle physics as we search for answers regarding the Higgs, dark matter, and other fundamental questions, many open questions remain at low energy scales. In particular because of its fundamentally non-perturbative character, understanding quantum chromodynamics (QCD) and strong interactions at low energies presents experimental and theoretical challenges. We review the experimental status of the light hadronic spectrum and some tools for low-energy theorists, including linear and non-linear $\sigma$ theories in the chiral approximation.
\end{abstract}
\maketitle
\section{Introduction}\label{I}
While much interest and resources have been directed towards the TeV energy scales within particle physics as we search for answers regarding the Higgs, dark matter, and other fundamental questions, many open questions remain at low energy scales. In particular because of its fundamentally non-perturbative character, understanding quantum chromodynamics (QCD) and strong interactions at low energies presents experimental and theoretical challenges. We review the experimental status of the light hadronic spectrum and some tools for low-energy theorists, focusing on linear and non-linear $\sigma$ theories in the chiral approximation.


\section{Low-energy QCD}\label{II}
\subsection{Experimental Background}
Below $2\,\mathrm{GeV}$, there exists a host of unresolved scalar hadronic resonances. While the higher energy vector and tensor states tend to be better understood, the scalar mesons are plagued with difficulties arising from wide resonance widths \& overlaps, cusps at mass thresholds, and the expectation of contamination by non-$\bar{q}q$ scalar states \cite{PDG2014}. Much work has been done to identify the scalar nonet from these unresolved resonances, to parallel the work done with the ground state vector, pseudoscalar, and tensor mesons [REF- VERIFY]. 
The masses of these scalar mesons seem to indicate a structure distinct from that of the vector or tensor mesons (see Figure \ref{}); the reversed mass hierarchy is consistent with a four-quark ($qq\bar{q}\bar{q}$) interpretation rather than a $q\bar{q}$ structure \cite{Fariborz2010,Jaffe1977}. 

The most experimentally complex case of these scalar mesons are the lightest scalar-isoscalar mesons $\left(I(J^{PC})= 0(0^{++})\right)$. 

\subsection{Symmetries}
We know now that because of colour confinement free quarks are not observed; historically, much of the investigation into hadronic structure had to be done implicitly. Much of our understanding of the strong interaction was developed before the notion of quarks and gluons was even conceptualized; by examining mathematical symmetries of the Lagrangians and equations of motion, deep insights can be made into the nature of matter. Emmy Noether's insight into how continuous symmetries in nature translate into conserved quantities was a revolutionary shift that has enormously impacted how we approach modern physics. She stated that...[MATHS HERE]. Her ideas form the foundation of many of the most important insights into particle physics to date.

As important to examine are the deviations from symmetries that we see in nature, also known as symmetry breaking. Symmetry breaking in quantum field theories manifests in three different ways:
\begin{itemize}
    \item Explicit symmetry breaking
    \item Spontaneous symmetry breaking
    \item Anomolous (quantum mechanical) symmetry breaking
\end{itemize}
QCD contains examples of all three of these. We'll examine the evidence and theory around spontaneous symmetry breaking, and follow that with a discussion about explicit symmetry breaking in QCD.

\subsubsection{Spontaneous Symmetry Breaking}
Formally, spontaneous symmetry breaking (SSB) in the Nambu-Goldstone realization is defined by the condition
\begin{equation}\label{eq:ssb}
    \bra{0} \left[ Q_{a},\mathcal{O}_{i}\right]\ket{0} = -(T_{a})_{ij}\bra{0}\mathcal{O}_{j}\ket{0}\neq 0,
\end{equation}
where $(T_{a})_{ij}$ is the generator associated with the continuous symmetry, and $\mathcal{O}_i$ are a set of field operators. Conceptually speaking, we say that spontaneous symmetry breaking occurs if the ground-state of the field $\mathcal{O}_{i}$ does not carry the same symmetries as the vacuum.
[EXAMPLES]

As we will discuss in the next sections, there is strong evidence to support the idea that SSB occurs in QCD. However, we do not see it manifest in the Lagrangian as it did in the examples discussed. This points to a dynamical picture of SSB within QCD, of which the mechanism is not entirely clear. 
\subsubsection{Nambu-Goldstone Bosons}
An important signal of spontaneous symmetry breaking in a theory is the appearance of massless scalar bosons.

For example, by examining the patterns of known mesons, [TABLE HERE], we can see a large disparity in mass between the lowest lying pseudoscalars and the vector mesons. 
We don't see nearly the same disparity between heavy quark analogues, so this points to something unique happening in the light quark sector.

\subsection{Chiral Approximation}
\subsection{Effective Theories}

\section{Linear $\sigma$ Theories}\label{IV}

The linear sigma model was introduced prior to the proposal and discovery of quarks in an effort to develop a theory of interactions between pions and nucleons. In their original work \cite{GellMann1960}, Gell-Mann and Levy
\subsection{Extended L$\sigma$M}
\subsection{Applications}
\section{Non-linear $\sigma$ Theories}\label{V}
\subsection{Chiral Perturbation Theory}
\subsection{Applications}

\section{Discussion}\label{VI}

\clearpage
%\bibliographystyle{h-physrev}
\bibliography{research}

\end{document}